% !TEX encoding = UTF-8
% !TEX TS-program = pdflatex
% !TEX root = ../tesi.tex

\cleardoublepage
\phantomsection
\pdfbookmark{Sommario}{Sommario}
\begingroup

\chapter*{Organizzazione del testo}

\begin{description}
  \item[{\hyperref[cap:processi-metodologie]{Il secondo capitolo}}] descrive ...
  
  \item[{\hyperref[cap:descrizione-stage]{Il terzo capitolo}}] approfondisce ...
  
  \item[{\hyperref[cap:analisi-requisiti]{Il quarto capitolo}}] approfondisce ...
  
  \item[{\hyperref[cap:progettazione-codifica]{Il quinto capitolo}}] approfondisce ...
  
  \item[{\hyperref[cap:verifica-validazione]{Il sesto capitolo}}] approfondisce ...
  
  \item[{\hyperref[cap:conclusioni]{Nel settimo capitolo}}] descrive ...
\end{description}

Riguardo la stesura del testo, relativamente al documento sono state adottate le seguenti convenzioni tipografiche:
\begin{itemize}
	\item gli acronimi, le abbreviazioni e i termini ambigui o di uso non comune menzionati vengono definiti nel glossario, situato alla fine del presente documento;
	\item per la prima occorrenza dei termini riportati nel glossario viene utilizzata la seguente nomenclatura: \emph{parola}\glsfirstoccur;
	\item i termini in lingua straniera o facenti parti del gergo tecnico sono evidenziati con il carattere \emph{corsivo}.
\end{itemize}

\endgroup			

\vfill


%**************************************************************
% Ringraziamenti
%**************************************************************
\cleardoublepage
\phantomsection
\pdfbookmark{Ringraziamenti}{ringraziamenti}


\bigskip

\begingroup
\let\clearpage\relax
\let\cleardoublepage\relax
\let\cleardoublepage\relax

\chapter*{Ringraziamenti}

\noindent \textit{Innanzitutto, vorrei esprimere la mia gratitudine al Prof. Gilberto Filè, relatore della mia tesi, per la disponibilità che mi ha offerto durante il periodo di stage e per i preziosi insegnamenti durante questi tre anni.}\\

\noindent \textit{Un ringraziamento speciale a Cesare d'Amico e a tutti i colleghi di WorkWave per la calorosa accoglienza fin dai primi giorni di stage. E naturalmente un sentito grazie a Matteo Ronchi, che ha saputo sempre guidarmi saggiamente durante questa esperienza e condividere pazientemente le sue preziose conoscenze.}\\

\noindent \textit{Vorrei inoltre dare un caloroso abbraccio ad Elisa per l'affetto e per avermi insegnato ad amare i dettagli della vita. Un sentito ringraziamento invece per tutti gli amici, sia di università che di vita, i quali mi hanno sopportato e tenuto compagnia in questo percorso.}\\

\noindent \textit{Desidero infine ringraziare con affetto i miei genitori per il costante sostegno ed incitamento nei miei progetti e nelle mie decisioni.}\\

\bigskip

\noindent\textit{\myLocation, \myTime}
\hfill \myName

\endgroup

