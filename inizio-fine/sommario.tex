% !TEX encoding = UTF-8
% !TEX TS-program = pdflatex
% !TEX root = ../tesi.tex

%**************************************************************
% Sommario
%**************************************************************
\cleardoublepage
\phantomsection
\pdfbookmark{Sommario}{Sommario}
\begingroup

\chapter*{Sommario}

Il presente documento è la relazione finale del lavoro svolto durante il periodo di stage del laureando Giovanni Jiayi Hu presso l'azienda WorkWave Italy Srl, della durata di 312 ore. \\

Lo scopo dello stage è stato l'esplorazione delle più recenti tecnologie web per la realizzazione di applicazioni web multi-finestra. A tale scopo è stato necessario eseguire un'attività di Ricerca \& Sviluppo (R\&D), testarne la loro maturità e integrare tali tecnologie nel prodotto \textbf{WorkWave Route Manager}. Quest'ultima applicazione offre ai propri clienti e permette di pianificare, dirigere, tracciare e analizzare le rotte dei  veicoli in tempo reale. \\

Nello specifico, l'obiettivo principale è poter estrarre porzioni d'interfaccia grafica dall'applicazione in una nuova pagina web autonoma, ma sincronizzata a livello di stato applicativo. L'integrazione ha avuto difatti lo scopo di permettere la visualizzazione della mappa Google Maps, ricca di rotte ed veicoli, su un monitor separato full-screen. \\

Lo sviluppo ha portato dunque alla nascita della libreria denominata \textbf{Stargate}, ispirato dall'omonima serie tv di fantascienza. Maggiori dettagli sugli obiettivi del progetto sono presentati alla sezione §\ref{presentazione-progetto}. \\

Sia \textit{Stargate} che \textit{Route Manager} sono basati su TypeScript, un linguaggio tipizzato che compila in JavaScript, ed utilizzano le librerie React 16 e Redux 4. Maggiori informazioni sulle tecnologie sono disponibili in Appendice §\ref{tecnologie} di questo documento.

%\vfill
%
%\selectlanguage{english}
%\pdfbookmark{Abstract}{Abstract}
%\chapter*{Abstract}
%
%\selectlanguage{italian}

\endgroup			

\vfill

