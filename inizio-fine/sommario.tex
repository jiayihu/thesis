% !TEX encoding = UTF-8
% !TEX TS-program = pdflatex
% !TEX root = ../tesi.tex

%**************************************************************
% Sommario
%**************************************************************
\cleardoublepage
\phantomsection
\pdfbookmark{Sommario}{Sommario}
\begingroup

\chapter*{Sommario}

Il presente documento è la relazione finale del lavoro svolto durante il periodo di stage del laureando Giovanni Jiayi Hu presso l'azienda WorkWave Italy Srl, della durata di 312 ore. \\

Lo scopo dello stage è stato l'esplorazione e l'apprendimento delle più recenti tecnologie web per la realizzazione di applicazioni web multi-finestra. A tale scopo è stato necessario eseguire un'attività di Ricerca \& Sviluppo (R\&D), testarne la loro maturità e realizzare un Proof of Concept che sfrutti tali tecnologie per poter estrarre porzioni di interfaccia grafica dall'applicazione principale in una nuova pagina autonoma, ma sincronizzata a livello di stato applicativo. \\

La prima fase delle attività ha portato dunque alla nascita della prima versione di una libreria battezzata col nome \textit{Stargate}, ispirato dall'ononima serie tv di fantascienza. \\

In secondo luogo è stato richiesta l'evoluzione e l'integrazione di Stargate nell'applicazione web in via di sviluppo denominata \textit{WorkWave Route Manager}. Quest'ultima è la nuova versione di uno dei prodotti principali che l'azienda offre ai propri clienti e permette di pianificare, dirigere, tracciare e analizzare le rotte dei propri veicoli in tempo reale. \\
L'integrazione ha avuto difatti l'obiettivo di permettere la visualizzazione della mappa Google Maps, ricca di rotte ed veicoli, su un monitor separato full-screen. \\

Sia \textit{Stargate} che \textit{Route Manager} sono basati su TypeScript, un linguaggio tipizzato che compila in JavaScript, ed utilizzano le librerie React 16 e Redux 4. In particolare \textit{Stargate} è usufruibile su qualsiasi applicazione web JavaScript, ma fornisce già le integrazioni per agevolarne l'uso con React e Redux.

%\vfill
%
%\selectlanguage{english}
%\pdfbookmark{Abstract}{Abstract}
%\chapter*{Abstract}
%
%\selectlanguage{italian}

\endgroup			

\vfill

