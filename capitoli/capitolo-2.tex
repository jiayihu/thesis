% !TEX encoding = UTF-8
% !TEX TS-program = pdflatex
% !TEX root = ../tesi.tex

%**************************************************************
\chapter{Processi e metodologie}
\label{cap:processi-metodologie}
%**************************************************************

Durante il periodo di stage, ho avuto l'opportunità di entrare in contatto con i processi aziendali e diversi strumenti a supporto del mio lavoro, di seguito illustrati.

\section{Accertamento di Qualità}

Il processo di Accertamento di qualità provvede a garantire che il prodotto software sia conforme alle aspettative di qualità desiderate. Nello specifico, durante il mio periodo di stage, sono venuto a contatto con le seguenti pratiche di sviluppo.

\subsection{Pull Request}

Una Pull Request è una proposta di modifica al repository effettuata su Github. Essa è obbligatoria per qualsiasi modifica e deve essere sempre realizzata tramite un git branch dedicato, avente un nome univoco e semantico rispetto alle modifiche proposte. \\

Lo scopo della Pull Request in WorkWave è favorire la discussione delle modifiche da parte del team e permetterne un'attenta ispezione prima ritenerla valida. Tuttavia essa è anche un'opportunità di apprendimento sia per chi esegue la review che per chi la riceve, in quanto entrambi hanno modo di apprendere diversi approcci allo stesso problema. \\

È stato inoltre spiegato che essa permette anche, nel lungo termine, di venire a conoscenza di problematiche nel processo di sviluppo e poterle migliorare. Ad esempio la continua segnalazione di norme di sintassi è un indice della necessità di introdurre uno strumento automatico per la formattazione del codice.

\section{Gestione della configurazione}

\subsection{Versionamento}

L'azienda WorkWave organizza il proprio codice sorgente all'interno di diverse repository Git raggruppate sotto l'organizzazione GitHub dell'azienda. In particolare è stato creato un repository dedicato al versionamento della prima versione della libreria \textit{Stargate} assieme al Proof of Concept. Successivamente il codice sorgente della libreria è stato direttamente integrato nel repository dell'applicazione \textit{Route Manager}.

\subsection{Ambiente di verifica}

\begin{figure}[H] 
  \centering 
  \includegraphics[width=0.5\columnwidth]{jenkins} 
  \caption{Jenkins}
\end{figure}

Il processo di verifica è il più automatizzato possibile tramite tools eseguiti automaticamente con Jenkins \url{https://jenkins.io/}. Lo stesso procedimento avviene ad ogni Pull Request proibendone l’accettazione se le verifiche non sono superate.

Inoltre sono presenti script automatici che permettono di rilasciare in ambiente di sviluppo, demo, testing e produzione attraverso l'interfaccia grafica dashboard di Jenkins.

\subsection{Ambiente di rilascio}

\begin{figure}[H] 
  \centering 
  \includegraphics[width=0.5\columnwidth]{clodfront} 
  \caption{Amazon Cloudfront}
\end{figure}

Il rilascio automatico eseguito da Jenkins porta al caricamento dell'applicazione \textit{Route Manager} su Amazon Cloudfront \url{https://aws.amazon.com/it/cloudfront/}, un servizio di Content Delivery Network (CDN) che permette di distribuire l'applicazione con latenza minima nelle diversi Paesi del mondo.

\section{Gestione di Progetto}

\subsection{Stand-up}

Il team si incontra quotidianamente per lo stand-up, un incontro informale senza durata prefissata, che permette ai vari membri di allinearsi reciprocamente sullo stato di avanzamento ed eventuali problematiche. In particolare, nel caso di WorkWave tale attività è indispensabile in quanto vi sono alcuni del team che lavorano in remoto o negli Stati Uniti.

Tutti i membri, non solo i programmatori, sono invitati a partecipare e ad esporre su cosa stiano lavorando ed potenziali criticità, permettendo anche di trasmettere maggiore consapevolezza e conoscenza del progetto ai diversi partecipanti.

\subsection{Ticketing}

\begin{figure}[H] 
  \centering 
  \includegraphics[width=1\columnwidth]{releaseplan_main} 
  \caption{CA Agile Central}
\end{figure}

L'azienda utilizza lo strumento CA Agile Central \url{https://www.ca.com/it/products/ca-agile-central.html} per la gestione delle attività.
