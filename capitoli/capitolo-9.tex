% !TEX encoding = UTF-8
% !TEX TS-program = pdflatex
% !TEX root = ../tesi.tex

%**************************************************************
\chapter{Valutazione retrospettiva}
\label{cap:valutazione}
%**************************************************************

Si riporta di seguito una valutazione retrospettiva dell'esperienza di stage, al fine di valutare ciò che è stato fatto, le esperienze apprese e trarre insegnamenti per il futuro.

\section{Raggiungimento degli obiettivi}

Durante lo sviluppo è emerso che alcuni requisiti quali "\texttt{O05 - Supporto a multi-sessione}" fossero estramente facili da implementare con le tecnologie descritte nei capitoli precedenti, mentre altre come "\texttt{O06 - Gestione della configurazione di Route Manager per supportare i widget}" sono state difficoltose quanto previsto o addirittura di più. \\

Tuttavia sia il lavoro di ricerca che il successivo sviluppo sono proseguiti bene nonostante qualche imprevisto lungo la strada ed entrambi le parti, azienda WorkWave e sottoscritto, siamo soddisfatti del risultati. Difatti tutti gli obiettivi obbligatori sono stati raggiunti con successo e vi è stato sufficiente tempo per il completamento anche di quelli desiderabili. \\

Prima del termine dell'esperienza, la libreria \textit{Stargate} è stata integrata correttamene nell'applicazione \textit{Route Manager} ed è attualmente già in uso in produzione da parte di alcuni clienti del prodotto. \\

\begin{table}[H]
\small
\begin{tabular}{ |p{1cm} |p{9cm}| p{2cm}|}
\hline
\textbf{ID} & \textbf{Descrizione} & \textbf{Stato} \\ \hline

\multicolumn{3}{|c|}{\textbf{Obbligatori}} \\ \hline

O01 & Supporto multi-finestra di componenti React & \textcolor{LimeGreen}{Completato} \\ \hline
O02 & Supporto per grandi moli di dati in continuo aggiornamento, dell'ordine di alcuni MByte & \textcolor{LimeGreen}{Completato} \\ \hline
O03 & Supporto a molteplici finestre in contemporanea, tutte sincronizzate rispetto all'applicazione padre & \textcolor{LimeGreen}{Completato} \\ \hline
O04 & Supporto per i browser moderni Google Chrome v56 e Firefox v38 & \textcolor{LimeGreen}{Completato} \\ \hline
O05 & Supporto a multi-sessione & \textcolor{LimeGreen}{Completato} \\ \hline
O06 & Gestione della configurazione di \textit{Route Manager} per supportare i widget & \textcolor{LimeGreen}{Completato} \\ \hline
O07 & Produzione della documentazione d'uso di \textit{Stargate} & \textcolor{LimeGreen}{Completato} \\ \hline
O08 & Utilizzo del linguaggio TypeScript v3 & \textcolor{LimeGreen}{Completato} \\ \hline

\multicolumn{3}{|c|}{\textbf{Desiderabili}} \\ \hline

D01 & Possibilità di eseguire il componente in una nuova pagina che risieda su un processo separato del sistema operativo & \textcolor{LimeGreen}{Completato} \\ \hline
D02 & Supporto prestazionale fino ad almeno 5 tabs simultanee & \textcolor{LimeGreen}{Completato} \\ \hline
D03 & Supporto librerie JavaScript non React, ad esempio Angular 2 & \textcolor{LimeGreen}{Completato} \\ \hline

\multicolumn{3}{|c|}{\textbf{Facoltativi}} \\ \hline

O04 & Supporto per i browsers Internet Explorer v11 e Edge v12 & \textcolor{Orange}{Non completato} \\ \hline

\end{tabular}
\caption{Tabella raggiungimento degli obiettivi}
\end{table}

\section{Resoconto dell'analisi dei rischi}

Si riporta di seguito anche il riscontro dei rischi analizzati ad inizio stage.

\begin{table}[H]
\small
\begin{tabular}{ |p{4.5cm} |p{1.75cm} |p{6.5cm}|}
\hline
\textbf{Descrizione} & \textbf{Verificato} & \textbf{Piano attuato} \\ \hline
\textbf{Difficoltà tecnologica}: vi è il rischio che lo stato dell'arte dello sviluppo web non consenta di aprire finestra come processi separati o che non sia possibile effettuare la comunicazione tra pagine diverse. & NO & Nessuno, la scrupolosa analisi iniziale delle tecnologie è stata fruttuosa. \\ \hline

\textbf{Difficoltà di integrazione}: vi è il rischio che non sia possibile integrare le tecnologie adottate nel contesto dell'applicazione \textit{Route Manager}, in quanto è già in sviluppo da oltre un anno e non progettata in partenza per essere multi-finestra. & SI & Grazie al supporto del tutor aziendale Matteo Ronchi sono state inviduate soluzioni dal buon compromesso che hanno permesso l'integrazione. I dettagli sono elaborati nel capitolo §\ref{cap:integrazione}. \\ \hline
\end{tabular}
\caption{Tabella dell'analisi dei rischi}
\end{table}

\section{Conoscenze acquisite}

Il periodo di stage è stato ricco sia di insegnamenti a livello di conoscenze tecniche che a livello di esperienze personali. Di seguito si riportano le principali skill acquisite.

\subsection{Ricerca \& Sviluppo}

Sebbene abbia già svolto attività R\&D in passato, questa è stata la prima esperienza in tal senso della durata di due mesi a tempo pieno. Ho avuto la fortuna di poter analizzare con calma lo stato tecnologico e cercare di non perdere alcuna informazione. Sono state difatte più volte piccoli informazioni apparentemente non inerenti che hanno fatto la differenza nella direzione presa dal progetto e addirittura a determinarne il successo o fallimento. \\

Ho osservato quanto sia importante, nella fase di ricerca tecnologica, non precludersi da alcuna informazione sebbene sembri a primo impatto solo in parte affina all'argomento d'interesse. Proprio data la natura sperimentale del progetto, è importante raccogliere maggiori informazioni possibili ed elaborarle per formare una strategia di sviluppo. \\

Esempio di questa esperienza è stato l'apprendimento del parametro di sicurezza \texttt{rel=noopener} per forzare l'uso di nuovi processi ad ogni finestra §\ref{forzare-processo}, dedotto per puro caso da un articolo in cui si accennava di esso unicamente per scopo di sicurezza senza menzionare riguardo a processi. \\

Inoltre ho appreso a mie spese che è bene non tralasciare di leggere il codice sorgente di alcuni progetti open-source, in quanto vi sono molte informazioni importanti assenti nella documentazione poiché magari ritenute dall'autore di utilità improbabile. \\

Infine è stato importante anche il tracciamento delle fonti e delle informazioni fin dal primo momento, a fine di documentare il processo di ricerca effettuato. Sarà possibile in tal modo capire come è stato raggiunto il risultato e quali altre possibili vie non sia state prese.

\subsection{Tecnologie apprese}

Tra le conoscenze puramente tecniche ho avuto modo di migliorare la mia conoscenza del \textit{type-level programming}, ovvero dell'uso avanzato dei tipi statici in TypeScript. In particolare ho approfondito l'uso degli \texttt{enums} e soprattutto dei \textit{conditional types} \footnote{\url{https://www.typescriptlang.org/docs/handbook/release-notes/typescript-2-8.html}}. \\

Inoltre ovviamente ho migliorato le mie conoscenze per quanto riguarda la comunicazione asincrona per messaggi, avendo ad esempio usato ampliamente \texttt{postMessage} e implementato da zero le chiamate RPC §\ref{rpc}. \\

Infine vi è stata anche l'opportunità di lavorare a livello avanzato con le API di Google Maps, quali ad esempio \textit{markers, disegno sulla mappa, polylines e polygons}.

\section{Valutazione personale}

Questa esperienza in WorkWave mi ha permesso di conoscere in primis persone fantastiche, non solo a livello professionale, e di imparare ad introdurmi in un team dove vi sono dei veri valori condivisi dai singoli. Lavorando in team è infatti fondamentale il rispetto del lavoro altrui e ad allo stesso tempo la capacità di saper ricevere critiche e migliorare il risultato progressivamente.

Inoltre la possibilità di condividere lo spirito verso nuove sfide e di risolvere un problema insieme ad un collega arricchisce ancor di più l'esperienza ed il rafforzamento dello spirito di appartenenza al team. Saper lavorare in team è indispensabile per perseguire un risultato di qualità, ma anche per poter continuare ad amare questo lavoro. \\

A livello tecnico ho invece potuto felicemente constastare l'aiuto fornito dalle nozioni apprese durante il corso di studi e comprendere l'importanza di una solida base per poter mirare in alto. \\

In conclusione l'esperienza in WorkWave è stata ricca sia a livello conoscitivo che umano e spero poter ripetere anche in futuro.
