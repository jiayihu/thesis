% !TEX encoding = UTF-8
% !TEX TS-program = pdflatex
% !TEX root = ../tesi.tex

%**************************************************************
\chapter{Tecnologie utilizzate}
%**************************************************************

\section{TypeScript}

\begin{figure}[H] 
  \centering 
  \includegraphics[width=0.25\columnwidth]{ts} 
  \caption{TypeScript}
\end{figure}

TypeScript è un linguaggio di programmazione open-source sviluppato e mantenuto da Microsoft. La sua sintassi è un super-set di JavaScript, ovvero qualsiasi sintassi di quest'ultima è valida anche in TypeScript ma non vice-versa. TypeScript aggiunge difatti la possibilità di avere tipi statici. \\

TypeScript è progettato per lo sviluppo di grandi applicazioni e compila in JavaScript, per cui può essere usato sia per l'esecuzione lato client che server (Node.js). \\

Il linguaggio fornisce i tipi statici attraverso \textit{type annotations} che attivano il controllo dei tipi a tempo di compilazione. La scelta è tuttavia opzionale ed è possibile scrivere normale codice JavaScript senza tipi. \\

\begin{lstlisting}
function add(left: number, right: number): number {
  return left + right;
}
\end{lstlisting}

I tipi statici possono anche essere esportati dopo la compilazione in file di dichiarazione separati, in modo da fornire le informazioni sui tipi a chi deve utilizzare la libreria, la quale sarà già stata compilata in JavaScript. \\

Sia la libreria \textit{Stargate} che l'applicazione \textit{WorkWave Route Manager} sono scritte in TypeScript 3.0, rilasciato ad agosto 2018.
