
%**************************************************************
% Acronimi
%**************************************************************
\renewcommand{\acronymname}{Acronimi e abbreviazioni}

\newacronym[description={\glslink{apig}{Application Program Interface}}]
    {api}{API}{Application Program Interface}

%**************************************************************
% Glossario
%**************************************************************
%\renewcommand{\glossaryname}{Glossario}

\newglossaryentry{apig}
{
    name=API,
    description={In informatica con il termine \emph{Application Programming Interface API} (ing. interfaccia di programmazione di un'applicazione) si indica ogni insieme di procedure disponibili al programmatore, di solito raggruppate a formare un set di strumenti specifici per l'espletamento di un determinato compito all'interno di un certo programma. La finalità è ottenere un'astrazione, di solito tra l'hardware e il programmatore o tra software a basso e quello ad alto livello semplificando così il lavoro di programmazione}
}

\newglossaryentry{cookies}
{
    name=cookies,
    description={Un HTTP cookie è un piccolo pezzo di informazione inviato da un server e salvato nel browser dell'utente durante la sua navigazione per l'identificazione di quest'ultimo}
}

\newglossaryentry{design-patterns}
{
    name=design-patterns,
    description={Si tratta di una descrizione o modello logico da applicare per la risoluzione di un problema che può presentarsi in diverse situazioni durante le fasi di progettazione e sviluppo del softwarex}
}

\newglossaryentry{framework}
{
    name=framework,
    description={Nello sviluppo software, è un'architettura logica di supporto (spesso un'implementazione logica di un particolare design pattern) su cui un software può essere progettato e realizzato, spesso facilitandone lo sviluppo da parte del programmatore.
    Un framework è definito da un insieme di classi astratte e dalle relazioni tra esse}
}

\newglossaryentry{JavaScript}
{
    name=JavaScript,
    description={Linguaggio dinamico ad alto livello che non fa uso di tipi. Viene molto usato nel web e nello sviluppo di applicazioni online. Il linguaggio non è da confondersi con il linguaggio Java, col quale condivide solo parte del nome}
}

\newglossaryentry{JSON}
{
    name=JSON,
    description={Un formato adatto all'interscambio di dati tra applicazioni client-server. Basato sul linguaggio JavaScript ma indipendente da esso}
}

\newglossaryentry{Pair Programming}
{
    name=Pair Programming,
    description={Pratica Agile nel quale due persone sono responsabili dello sviluppo del codice e operano insieme per realizzarlo}
}

\newglossaryentry{React}
{
    name=React,
    description={Una libreria JavaScript open-source per l'implementazione di interfacce utenti}
}

\newglossaryentry{Redux}
{
    name=Redux,
    description={Una libreria JavaScript open-source per la gestione dello stato applicativo}
}

\newglossaryentry{singleton}
{
    name=singleton,
    description={Il singleton è un design pattern creazionale che ha lo scopo di garantire che venga creata una e una sola istanza di una determinata classe, e di fornire un punto di accesso globale a tale istanza}
}

\newglossaryentry{sprint}
{
    name=sprint,
    description={Lo sprint è un'unità di base dello sviluppo Agile ed è di durata fissa, generalmente da una a quattro settimane}
}

\newglossaryentry{standup}
{
    name=standup,
    description={Un incontro informale della pratica Agile senza durata prefissata, che permette ai vari membri di allinearsi reciprocamente sullo stato di avanzamento ed eventuali problematiche}
}

\newglossaryentry{thread}
{
    name=thread,
    description={Un thread o thread di esecuzione, in informatica, è una suddivisione di un processo in due o più filoni o sottoprocessi che vengono eseguiti concorrentemente da un sistema di elaborazione monoprocessore (multithreading) o multiprocessore}
}

\newglossaryentry{TypeScript}
{
    name=TypeScript,
    description={Linguaggio open-source che estende la sintassi di JavaScript e compila in quest'ultimo. È dunque possibile scrivere in JavaScript ECMAScript 2015 con in aggiunta il supporto alla tipizzazione statica}
}
